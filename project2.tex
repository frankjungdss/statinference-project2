\documentclass[legalpaper]{article}
\usepackage{lmodern}
\usepackage{amssymb,amsmath}
\usepackage{ifxetex,ifluatex}
\usepackage{fixltx2e} % provides \textsubscript
\ifnum 0\ifxetex 1\fi\ifluatex 1\fi=0 % if pdftex
  \usepackage[T1]{fontenc}
  \usepackage[utf8]{inputenc}
\else % if luatex or xelatex
  \ifxetex
    \usepackage{mathspec}
    \usepackage{xltxtra,xunicode}
  \else
    \usepackage{fontspec}
  \fi
  \defaultfontfeatures{Mapping=tex-text,Scale=MatchLowercase}
  \newcommand{\euro}{€}
\fi
% use upquote if available, for straight quotes in verbatim environments
\IfFileExists{upquote.sty}{\usepackage{upquote}}{}
% use microtype if available
\IfFileExists{microtype.sty}{%
\usepackage{microtype}
\UseMicrotypeSet[protrusion]{basicmath} % disable protrusion for tt fonts
}{}
\usepackage[margin=1in]{geometry}
\usepackage{color}
\usepackage{fancyvrb}
\newcommand{\VerbBar}{|}
\newcommand{\VERB}{\Verb[commandchars=\\\{\}]}
\DefineVerbatimEnvironment{Highlighting}{Verbatim}{commandchars=\\\{\}}
% Add ',fontsize=\small' for more characters per line
\usepackage{framed}
\definecolor{shadecolor}{RGB}{248,248,248}
\newenvironment{Shaded}{\begin{snugshade}}{\end{snugshade}}
\newcommand{\KeywordTok}[1]{\textcolor[rgb]{0.13,0.29,0.53}{\textbf{{#1}}}}
\newcommand{\DataTypeTok}[1]{\textcolor[rgb]{0.13,0.29,0.53}{{#1}}}
\newcommand{\DecValTok}[1]{\textcolor[rgb]{0.00,0.00,0.81}{{#1}}}
\newcommand{\BaseNTok}[1]{\textcolor[rgb]{0.00,0.00,0.81}{{#1}}}
\newcommand{\FloatTok}[1]{\textcolor[rgb]{0.00,0.00,0.81}{{#1}}}
\newcommand{\CharTok}[1]{\textcolor[rgb]{0.31,0.60,0.02}{{#1}}}
\newcommand{\StringTok}[1]{\textcolor[rgb]{0.31,0.60,0.02}{{#1}}}
\newcommand{\CommentTok}[1]{\textcolor[rgb]{0.56,0.35,0.01}{\textit{{#1}}}}
\newcommand{\OtherTok}[1]{\textcolor[rgb]{0.56,0.35,0.01}{{#1}}}
\newcommand{\AlertTok}[1]{\textcolor[rgb]{0.94,0.16,0.16}{{#1}}}
\newcommand{\FunctionTok}[1]{\textcolor[rgb]{0.00,0.00,0.00}{{#1}}}
\newcommand{\RegionMarkerTok}[1]{{#1}}
\newcommand{\ErrorTok}[1]{\textbf{{#1}}}
\newcommand{\NormalTok}[1]{{#1}}
\usepackage{longtable,booktabs}
\usepackage{graphicx}
\makeatletter
\def\maxwidth{\ifdim\Gin@nat@width>\linewidth\linewidth\else\Gin@nat@width\fi}
\def\maxheight{\ifdim\Gin@nat@height>\textheight\textheight\else\Gin@nat@height\fi}
\makeatother
% Scale images if necessary, so that they will not overflow the page
% margins by default, and it is still possible to overwrite the defaults
% using explicit options in \includegraphics[width, height, ...]{}
\setkeys{Gin}{width=\maxwidth,height=\maxheight,keepaspectratio}
\ifxetex
  \usepackage[setpagesize=false, % page size defined by xetex
              unicode=false, % unicode breaks when used with xetex
              xetex]{hyperref}
\else
  \usepackage[unicode=true]{hyperref}
\fi
\hypersetup{breaklinks=true,
            bookmarks=true,
            pdfauthor={Frank Jung},
            pdftitle={Analysis of Tooth Growth},
            colorlinks=true,
            citecolor=blue,
            urlcolor=blue,
            linkcolor=magenta,
            pdfborder={0 0 0}}
\urlstyle{same}  % don't use monospace font for urls
\setlength{\parindent}{0pt}
\setlength{\parskip}{6pt plus 2pt minus 1pt}
\setlength{\emergencystretch}{3em}  % prevent overfull lines
\setcounter{secnumdepth}{5}

%%% Use protect on footnotes to avoid problems with footnotes in titles
\let\rmarkdownfootnote\footnote%
\def\footnote{\protect\rmarkdownfootnote}

%%% Change title format to be more compact
\usepackage{titling}

% Create subtitle command for use in maketitle
\newcommand{\subtitle}[1]{
  \posttitle{
    \begin{center}\large#1\end{center}
    }
}

\setlength{\droptitle}{-2em}
  \title{Analysis of Tooth Growth}
  \pretitle{\vspace{\droptitle}\centering\huge}
  \posttitle{\par}
  \author{Frank Jung}
  \preauthor{\centering\large\emph}
  \postauthor{\par}
  \date{}
  \predate{}\postdate{}



\begin{document}

\maketitle


{
\hypersetup{linkcolor=black}
\setcounter{tocdepth}{2}
\tableofcontents
}
\section{Synopsis}\label{synopsis}

In this project we will analyse the
\href{https://stat.ethz.ch/R-manual/R-devel/library/datasets/html/ToothGrowth.html}{ToothGrowth}
dataset to compare odontoblast growth by supplement and dosage. The
question we want to answer is: Does the delivery method of supplements
effect mean
\href{https://en.wikipedia.org/wiki/Odontoblast}{odontoblast} growth?
i.e.~Is there any measurable difference in means of growth between these
supplements and dosages?

\section{Exploratory Analysis}\label{exploratory-analysis}

First, let us explore the data:

\begin{Shaded}
\begin{Highlighting}[]
\CommentTok{# for convenience save ToothGrowth into data frame tg}
\NormalTok{tg <-}\StringTok{ }\KeywordTok{data.frame}\NormalTok{(ToothGrowth)}
\NormalTok{tg$dose <-}\StringTok{ }\KeywordTok{factor}\NormalTok{(tg$dose)}
\KeywordTok{str}\NormalTok{(tg)}
\end{Highlighting}
\end{Shaded}

\begin{verbatim}
## 'data.frame':    60 obs. of  3 variables:
##  $ len : num  4.2 11.5 7.3 5.8 6.4 10 11.2 11.2 5.2 7 ...
##  $ supp: Factor w/ 2 levels "OJ","VC": 2 2 2 2 2 2 2 2 2 2 ...
##  $ dose: Factor w/ 3 levels "0.5","1","2": 1 1 1 1 1 1 1 1 1 1 ...
\end{verbatim}

\begin{Shaded}
\begin{Highlighting}[]
\KeywordTok{table}\NormalTok{(tg[tg$supp ==}\StringTok{ "OJ"}\NormalTok{, }\KeywordTok{c}\NormalTok{(}\StringTok{"dose"}\NormalTok{)])}
\end{Highlighting}
\end{Shaded}

\begin{verbatim}
## 
## 0.5   1   2 
##  10  10  10
\end{verbatim}

\begin{Shaded}
\begin{Highlighting}[]
\KeywordTok{table}\NormalTok{(tg[tg$supp ==}\StringTok{ "VC"}\NormalTok{, }\KeywordTok{c}\NormalTok{(}\StringTok{"dose"}\NormalTok{)])}
\end{Highlighting}
\end{Shaded}

\begin{verbatim}
## 
## 0.5   1   2 
##  10  10  10
\end{verbatim}

\begin{Shaded}
\begin{Highlighting}[]
\KeywordTok{range}\NormalTok{(tg$len)}
\end{Highlighting}
\end{Shaded}

\begin{verbatim}
## [1]  4.2 33.9
\end{verbatim}

Below is a violin plot by supplement and dose:

\begin{center}\includegraphics{figure/exploreplot-1} \end{center}

In summary, the ToothGrowth dataset is small, with:

\begin{itemize}
\itemsep1pt\parskip0pt\parsep0pt
\item
  10 observations of odontoblast length (microns) for each dose
\item
  3 distinct dosage levels (milligrams)
\item
  2 supplement types (VC = Vitamin C and OJ = Orange Juice)
\end{itemize}

The data suggests that there may be observable differences with lower
dosages.

\section{Assumptions}\label{assumptions}

In this analysis we assume that:

\begin{itemize}
\itemsep1pt\parskip0pt\parsep0pt
\item
  the observations are independent
  (\href{http://jn.nutrition.org/content/33/5/491.full.pdf}{different
  groups of guinea-pigs})
\item
  the data is assumed to be normally distributed
  (\href{https://en.wikipedia.org/wiki/Normal_distribution\#/Approximate_normality}{likely
  as this is a biological measurement})
\end{itemize}

\section{Test Hypothesis}\label{test-hypothesis}

Our null hypothesis is that there is no difference in odontoblast mean
lengths between supplement types for each dosage. That is:
\(H_0 : \mu_{oj} = \mu_{vc}\) with an alternative hypothesis:
\(H_a : \mu_{oj} \ne \mu_{vc}\). To test this we will calculate a 95\%
confidence interval, using a two-sided T test of independent groups,
with variances assumed to be unequal.

For each dose, compare mean lengths from each supplement. Show just the
95\% confidence interval and p-value:

\begin{Shaded}
\begin{Highlighting}[]
\KeywordTok{by}\NormalTok{(tg, tg$dose, function (x) }\KeywordTok{t.test}\NormalTok{(len ~}\StringTok{ }\NormalTok{supp, x)[}\KeywordTok{c}\NormalTok{(}\StringTok{"conf.int"}\NormalTok{, }\StringTok{"p.value"}\NormalTok{)])}
\end{Highlighting}
\end{Shaded}

\begin{verbatim}
## tg$dose: 0.5
## $conf.int
## [1] 1.719057 8.780943
## attr(,"conf.level")
## [1] 0.95
## 
## $p.value
## [1] 0.006358607
## 
## -------------------------------------------------------- 
## tg$dose: 1
## $conf.int
## [1] 2.802148 9.057852
## attr(,"conf.level")
## [1] 0.95
## 
## $p.value
## [1] 0.001038376
## 
## -------------------------------------------------------- 
## tg$dose: 2
## $conf.int
## [1] -3.79807  3.63807
## attr(,"conf.level")
## [1] 0.95
## 
## $p.value
## [1] 0.9638516
\end{verbatim}

Note that the confidence interval for supplement dosages of 0.5 and 1.0
milligrams does not contain 0. In addition, the p-values for these
observations are much less than 5\%. Compare this to the confidence
interval for supplement dosages of 2.0 milligrams, which does contain 0
and the p-value is above 96\%.

\section{Conclusion}\label{conclusion}

We are testing the null hypothesis that odontoblast mean lengths are the
same, regardless of the supplement type or dosage:
\(H_0 : \mu_{oj} = \mu_{vc}\). The alternative hypothesis is that they
are not equal: \(H_a : \mu_{oj} \ne \mu_{vc}\).

From the \href{https://en.wikipedia.org/wiki/P-value}{p values} and
\href{https://en.wikipedia.org/wiki/Confidence_interval}{confidence
intervals} we can conclude:

\begin{longtable}[c]{@{}cl@{}}
\toprule
Dosage (milligrams) & Comparison of mean lengths between
supplements\tabularnewline
\midrule
\endhead
0.5 & We reject the null hypothesis\tabularnewline
1.0 & We reject the null hypothesis\tabularnewline
2.0 & We fail to reject the null hypothesis\tabularnewline
\bottomrule
\end{longtable}

So, it appears that effects on odontoblast mean length diminishes
between supplement types at higher dosages.

\section{References}\label{references}

\begin{itemize}
\itemsep1pt\parskip0pt\parsep0pt
\item
  \href{http://jn.nutrition.org/content/33/5/491.full.pdf}{Crampton, E.
  W. (1947) The growth of the odontoblast of the incisor teeth as a
  criterion of vitamin C intake of the guinea pig. The Journal of
  Nutrition.}
\item
  \href{https://en.wikipedia.org/wiki/Normal_distribution}{Normal
  Distribution. Wikipedia}
\item
  \href{https://en.wikipedia.org/wiki/Odontoblast}{Odontoblast.
  Wikipedia}
\item
  \href{https://stat.ethz.ch/R-manual/R-devel/library/datasets/html/ToothGrowth.html}{ToothGrowth
  dataset}
\end{itemize}

\(\hfill \square\)

\end{document}
